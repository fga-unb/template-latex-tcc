\begin{resumo}

    O InterSCity é uma plataforma de cidades inteligentes baseado em uma
    arquitetura de microsserviços, e tem como objetivo suportar aplicações de
    cidades inteligentes através de serviços reutilizáveis, interoperáveis e
    escaláveis. Contudo, o uso de ferramentas adequadas no
    processamento de seus dados ainda não se faz presente, sendo um obstáculo
    em cenários de maior massa de dados. Este trabalho tem como objetivos o
    projeto e desenvolvimento de um serviço de processamento de dados que
    permita ao InterSCity oferecer novos serviços a partir do uso de grande
    massas de dados, utilizando principalmente \textit{Big Data}, tecnologia chave para
    cidades inteligentes. A partir da adoção do estado da arte em arquiteturas
    de processamento de dados em conjunto com tecnologias atuais de \textit{Big Data},
    esperamos que o InterSCity consiga suportar o desenvolvimento de aplicações
    mais sofisticadas para cidades inteligentes.

 \vspace{\onelineskip}
 \noindent
 \textbf{Palavras-chaves}: Cidades Inteligentes, Big Data, Arquitetura Kappa.
\end{resumo}
