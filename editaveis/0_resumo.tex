\begin{resumo}

O InterSCity é uma plataforma de cidades inteligentes que tem como
    diferencial o foco em reuso e interoperabilidade, se destacando em
    relação as plataformas atuais.
Contudo, o uso de ferramentas adequadas no processamento de seus dados ainda
    não se faz presente, o que pode impossibilitar o uso de dados em larga escala.
Este trabalho tem então como objetivos o desenho e desenvolvimento de uma
    camada de processamento de dados que permita ao InterSCity o uso de dados em
    larga escala, utilizando principalmente Big Data, uma tecnologia chave para
    cidades inteligentes.
Com uma camada de processamentos que utilize tecnologias de ponta e o estado
    da arte em arquiteturas de processamento, é esperado que o InterSCity
    atinja o que precise para se firmar como a melhor plataforma de cidades
    inteligentes.

 \vspace{\onelineskip}
    
 \noindent
 \textbf{Palavras-chaves}: cidades inteligentes, big data, arquitetura kappa, arquitetura lambda
\end{resumo}
