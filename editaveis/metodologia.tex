  \chapter[Metodologia]{Metodologia}

  Este capítulo é destinado à apresentar toda a metodologia aplicada nesta
  pesquisa. A metodologia está sendo baseada na definição proposta por Silva e Menezes, que afirmam que toda pesquisa
  tem como objetivo encontrar respostas para hipóteses propostas
  \cite{da2005metodologia}

  Com o intuito de classificar a pesquisa sendo desevolvida, observou-se que a
  mesma pode ser enquadrada como aplicada, quantitativa, exploratória, bibliográfica e
  experimental \cite{da2005metodologia}.


  \section{Trabalhos relacionados}

  Durante a pesquisa deste trabalho, encontrou-se alguns trabalhos que também estudam recomendação baseada em contexto, sendo o tempo um dos
  principais atributos do contexto. Em \cite{lee2008time}, o tempo é usado para gerar avaliação implícita para um histórico de
  compras de diversos usuários. Tal abordagem foi comparada com um modelo de recomendação colaborativa tradicional, usando avaliação explícita dos usuários,
  sem considerar o tempo. Segundo a pesquisa, percebeu-se um aumento de precisão
  quando o tempo foi usado como parâmetro do perfil de usuário.

  Outra pesquisa relacionada foi a de \cite{ding2005time}. Nesta pesquisa, um
  sistema de recomendação colaborativo sensível ao tempo foi criado. O sistema proposto usava uma função de
  decaimento exponencial para priorizar itens recentemente avaliados e diminuir o peso de itens avaliados a mais tempo na recomendação.
  Quando comparado a um sistema de
  recomendação colaborativa clássico, percebeu-se um aumento de acurácia para o recomendador sensível ao tempo.

  Além da comparação de sistema de recomendação, foi também encontrado um trabalho que propõe um modelo diferente do clássico uso do tempo em recomendação, como o usado em
  \cite{ding2005time}. Este modelo foi proposto no artigo de \cite{basile2015modeling}, onde basicamente itens mais antigos não tem seu peso puramente descartado, e sim caso
  seu conteúdo seja diferente dos novos itens sendo recomendados. Baseado nos
  testes realizados, percebeu-se que criar o modelo de usuário dessa forma apresenta vantagens sobre
  a forma mais padrão de se usar o tempo para recomendação. Entretanto, o trabalho ainda precisa ser testado em bancos de dados de maior escala para poder aferir melhor
  seus resultados

  Por fim, um famoso trabalho que usa o contexto de tempo para recomendação se dá na pesquisa \cite{koren2010collaborative}, que apresenta o modelo de recomendação
  vencedor do concurso criado pelo \textit{Netflix} para aumentar a acurácia de suas recomendações de filme. A pesquisa citada usa como diferencial a análise do tempo da avaliação
  dos filmes realizados pelos usuários em um modelo similar a pesquisa de \cite{basile2015modeling}


  \section{Planejamento da pesquisa}

  Tendo como base os conceitos de recomendação apresentados e a forma dinâmica com
  que usuários instalam e usam novos pacotes em seus próprios sistemas, esta
  pesquisa visa então responder a seguinte questão problema:

  \begin{center}
  \textit{É possível melhorar a recomendação baseada em conteúdo de pacotes para
  usuários de sistemas GNU/Linux levando em consideração o contexto temporal de
  uso dos seus pacotes já instalados ?}
  \end{center}

  Vale ressaltar que para responder esta questão problema, um conjunto de
  hipóteses foi criada juntamente com um fluxo de trabalho, conforme pode ser visto na
  Figura \ref{fig:planejamento_pesquisa}. As hipóteses e o fluxo definido serão
  explicados nas seções à seguir.

  \begin{figure}[h]
    \centering
    \includegraphics[width=0.9\textwidth]{figuras/planejamento_pesquisa.eps}
    \caption{Processo usado para realização das atividades dessa pesquisa}
    \label{fig:planejamento_pesquisa}
  \end{figure}

  \subsection{Hipóteses} \label{sec:hipoteses}

  Considerando a questão problema definida, as seguintes hipóteses foram definidas
  como forma de validação:

  \begin{itemize} \item \textit{\textbf{h1:Ao aumentar o peso de pacotes
  recentemente usados no sistema, o perfil do usuário gerado será mais preciso,
  ocasionando assim uma recomendação mais apropriada.}} Ao usar o tempo de uso dos
  pacotes no sistema, e priorizar os pacotes mais recentemente usados na criação
  do perfil, acredita-se que a mesma irá produzir um perfil mais
  consistente do usuário, gerando assim recomendações mais efetivas.\item
  \textit{\textbf{h2: O perfil de usuário gerado por aprendizado de máquina aplicado a uma
  estratégia de recomendação irá prover melhor resultados que uma abordagem
  determinística para criação do perfil.}}
  Considerando os dados presentes em pacotes Debian que podem ser
  usados para prover a recomendação, pode-se entender que o uso de uma abordagem
  por aprendizado de máquina irá prover melhores resultados, devido à possível
  quantidade de atributos que podem caracterizar um pacote.
  \end{itemize}

  \subsection{Seleção do projeto}

  Para validar as hipóteses levantadas, faz-se necessário escolher um projeto de
  software livre que implemente um sistema de recomendação de pacotes
  para sistemas GNU/Linux e que não use nenhuma informação de contexto em suas
  recomendações, de forma a permitir a comparação das abordagens propostas com as
  já implementadas pelo projeto.

  \subsection{Descrição do projeto selecionado}

  O projeto selecionado que atendia os requisitos propostos para a pesquisa foi o
  \textit{AppRecommender}, um sistema de recomendação de aplicativos GNU/Linux,
  cuja principal característica é recomendar aplicações,
  onde “dada a lista de programas instalados em determinado sistema, o recomendador retorna uma lista de aplicativos
  sugeridos, que supostamente são aplicativos de potencial interesse para os usuários daquele sistema”
  \cite{araujo2011apprecommender}. A Figura \ref{fig:apprecommender_execucao}
  mostra o funcionamento do \textit{AppRecommender} no terminal de um usuário.

  \begin{figure}[h]
    \centering
    \includegraphics[width=0.9\textwidth]{figuras/apprecommender_execucao.eps}
    \caption{Exemplo de execução da aplicação \textit{AppRecommender}}
    \label{fig:apprecommender_execucao}
  \end{figure}

  O \textit{AppRecommender} possui três categorias para estratégias de recomendação, onde cada
  estratégia é uma forma diferente de se obter uma lista de aplicativos sugeridos
  através de uma lista de pacotes instalados no sistema. Essas categorias são: baseadas em conteúdo,
  colaborativas, e híbridas. Segue abaixo as especificações de cada categoria de estratégia:

  \begin{itemize}
      \item \textit{\textbf{Baseadas em conteúdo:}} Esse tipo de categoria de estratégia
          visa montar um perfil de usuário através da lista dos pacotes instalados no sistema,
          onde o mesmo consiste em uma lista composta por \textit{Debtags} ou termos contidos nas descrições de pacotes que são
          definidas pelo algoritmo de estratégia escolhida. Através dessa lista, a base de dados
          com todos os pacotes é consultada e o resultado da recomendação são os pacotes mais
          relevantes para o perfil criado.
      \item \textit{\textbf{Colaborativas:}} Essa estratégia de recomendação visa
          a sugestão de novos pacotes via análise de outros usuários com perfis
          semelhantes ao do usuário, ou seja, tal estratégia visa recomendar novos
          pacotes de acordo com os pacotes instalados dos vizinhos mais próximos
          ao perfil do usuário.
      \item \textit{\textbf{Híbridas:}} Essa categoria
          de estratégia utiliza as duas anteriores, onde há formas diferentes de combiná-las
          afim de se obter diferentes recomendações.
  \end{itemize}

  Para as recomendações colaborativas e híbridas, o \textit{AppRecommender}
  utiliza uma sistemática para aumentar o peso de pacotes baseados no seu tempo.
  Entretanto esse não é o caso para recomendações puramente baseadas em conteúdo,
  fazendo assim que o mesmo se encaixe nos pré-requisitos levantados para o
  projeto.

  O projeto também utiliza tanto estratégias offline e testes de usuário para
  validar as recomendações implementadas, sendo que as estratégias offline têm como
  base a análise de curvas ROC, enquanto os testes de usuário visam analisar
  principalmente a precisão e novidade das recomendações.

  A licença do \textit{AppRecommender} é GNU GPL e o projeto está disponível em repositório
  público, permitindo assim que qualquer pessoa possa visualizar o código fonte e
  contribuir com o projeto. O código fonte da aplicação pode ser encontrado no
  repositório de projetos Github \footnote{\url{https://github.com/tassia/AppRecommender}}.

  As modificações propostas nesta pesquisa foram realizadas em uma cópia local do
  mesmo, sendo essa mantida pelos pesquisadores deste trabalho no Github \footnote{\url{https://github.com/TCC-AppRecommender/AppRecommender}}

  \section{Coleta de dados}

  Nesta seção, serão descritos os dados necessários para a execução do
  \textit{AppRecommender}, juntamente com os dados necessários para a coleta de
  informação contextual de tempo dos pacotes do usuário durante a
  recomendação.

  \subsection{AppRecommender}

  Para o projeto selecionado, observou-se a necessidade de diversos tipos
  de dados, que podem ser vistos na Tabela
  \ref{tab:dados_apprecommender}. Tal tabela mostra os dados usados no
  \textit{AppRecommender} e como os mesmos podem ser obtidos.

  \begin{table}[h]
  \centering
  \caption{Dados necessários para execução do AppRecommender}
  \label{tab:dados_apprecommender}
  \begin{tabularx}{\textwidth}{|X|X|X|}
  \hline
  \rowcolor[HTML]{EFEFEF}
  {\textbf{Dados}} & {\textbf{Contexto}} & {\textbf{Coleta}} \\ \hline
  Lista de nome dos pacotes válidos &
  Esta lista irá ser usada na construção de um índice de busca
  usado para buscar os pacotes que podem ser recomendados &
  Coleta automatizada via script \\
  \hline
  Índice de busca para os pacotes do usuário &
  Usado tanto para estratégias de recomendação por conteúdo,
  colaborativas e híbridas &
  Criação automatizada via script\\
  \hline
  Debtags válidas &
  Usado para filtrar os pacote que serão usados na
  recomendação &
  Coleta automatizada via script\\
  \hline
  Submissões do popularity contest &
  Usada para criar um índice de busca contendo as submissões do
  popularity-contest de todos os usuários que possuem uma submissão &
  Dados precisam ser manualmente coletados \\
  \hline
  Indíce de busca para submissões do popularity contest &
  Usado tanto para a realização de recomendações colaborativas e
  híbridas, assim como para a execução das coletas de opinião usados
  para avaliar as soluções desenvolvidas. &
  Criação automatizada via script \\
  \hline
  \end{tabularx}
  \end{table}

  Conforme pode ser observado na Tabela \ref{tab:dados_apprecommender}, muito dos dados
  necessários para a execução do \textit{AppRecommender} podem ser obtidos de forma automatizada
  por scripts de coleta. Entretanto, este não é o caso para as submissões do
  \textit{popularity-contest}. Isso se dá pelo fato da aplicação não disponibilizar as submissões
propriamente ditas, e sim as análises estatísticas feitas em cima das mesmas. Para o
\textit{AppRecommender}, tais dados foram obtidos por intermédio de um
\textit{Debian Developer}, sendo que algumas questões relacionadas a privacidade dos usuários se tornou necessária.
\cite{araujo2011apprecommender}. Tal problemática dificulta bastante o uso das
estratégias colaborativas do \textit{AppRecommender}, pois as mesmas dependem desse banco de dados
de submissões do \textit{popularity-contest}.

Uma observação também importante se dá na forma como a lista de pacotes válidos
é obtida, pois a mesma é obtida conforme as configurações do \textit{sources.list} do usuário. Este arquivo é responsável
por dizer ao gerenciador de pacotes onde o mesmo irá encontrar os arquivos binários ou
arquivos fonte dos pacote. A Figura \ref{fig:source_list} mostra o conteúdo deste arquivo:
\\

\begin{figure}[h]
  \centering
  \includegraphics[width=0.9\textwidth]{figuras/source_list.eps}
  \caption{Arquivo source.list de um sistema Debian}
  \label{fig:source_list}
\end{figure}

Conforme visto na Figura \ref{fig:source_list}, pode-se ver que a mesma indica
pelo rótulo \textit{deb} onde irá puxar os arquivos binários e pelo
\textit{deb-src} onde o mesmo irá puxar os arquivos fonte. Além disso, após o
link de cada um dos rótulos, existem também atributos, sendo eles a distribuição
do usuário e os componentes da mesma. Na figura, pode-se ver que o usuário está
usando a distribuição \textit{unstable} do Debian. Além disso, o Debian pode possuir três
tipos de pacotes, sendo eles:

\begin{itemize}
    \item \textbf{Main:} pacotes totalmente software livre;
    \item \textbf{Contrib:} pacotes software livre, mas que dependem de
        softwares não livres;
    \item \textbf{Non-free:} pacotes que não são software livre;
\end{itemize}

Sendo assim, como a lista de pacotes válidos é obtida localmente, a mesma irá ser baseada tanto na
distribuição que o usuário usa, como dos componentes que o mesmo possui no seu
\textit{sources.list}.

Por fim, vale ressaltar que foi identificado que as recomendações
colaborativas dependiam também de um serviço fornecido pela infra-estrutura do
Debian, chamado de \textit{Debian Data Export}, que era usado para facilitar a criação de
queries no \textit{Ultimate Debian Database}. Entretanto, este serviço se encontra
atualmente fora do ar, e, por isso, não é considerado na execução da pesquisa.

\subsection{Usuários} \label{sec:coleta_dados_usuario}

De forma a melhor avaliar como o contexto temporal está sendo usado pelas novas
estratégias, decidiu-se junto com a coleta de opinião dos usuários, coletar
as seguintes informações:

\begin{itemize}
    \item \textit{\textbf{Pacotes instalados:}} Necessário para se ter conhecimento dos pacotes que o usuário utiliza;
    \item \textit{\textbf{Pacotes manualmente instalados:}} Estes serão coletados para filtrar as preferências do usuário, para que quando comparado aos pacotes instalados se possa diferenciar pacotes que são automaticamente instalados dos que foram manualmente instalados;
    \item \textit{\textbf{Tempo de acesso e de modificação dos pacotes:}} Estes dados serão usados afim de obter
        informação quanto à classificação do pacote em relação ao tempo que este foi utilizado;
    \item \textit{\textbf{Caminho do binário de cada pacote:}} Dados coletados afim de se saber qual o binário responsável pela execução do pacote;
    \item \textit{\textbf{Versão do Kernel:}} Necessário para agrupar as recomendações e preferências entre usuários com mesmo kernel;
    \item \textit{\textbf{Distribuição do sistema:}} Necessário para agrupar as recomendações e preferências entre usuários com a mesma distribuição.
    \item \textit{\textbf{Submissão do popularity-contest:}} Informação que é
        utilizada para permitir a comparação do contexto temporal implementado
        na pesquisa e o usado pelo \textit{popularity-contest}.
\end{itemize}

O Apêndice \ref{ape:coleta_de_dados} explica como os dados são coletados.

\section{Análise dos Dados}

Este seção irá mostrar a análise dos dados que são usados para prover uma
recomendação propriamente dita. Além disso, também serão apresentados os novos dados
que esta pesquisa se propõe a anexar as atuais recomendações.

\subsection{Recomendação baseada em conteúdo}

Para recomendações baseadas em conteúdo, é necessário analisar tanto os dados
que são usados para construir o perfil do usuário quanto para realizar a
recomendação propriamente dita.

Para a construção do perfil do usuário, os pacotes instalados pelo usuário são
usados. Sendo assim, nenhum pacote possui uma avaliação explícita no sistema,
ou seja, a avaliação do item de recomendação nesse caso consiste apenas no fato
do usuário possuir o pacote instalado em seu sistema.

Para analisar um pacote Debian, pode-se observar o arquivo \textit{control} que
está presente em todo pacote Debian. Este arquivo
armazena todas as informações que são usadas para a construção do perfil do
usuário. A Figura \ref{fig:control_pacote} mostra um exemplo desse arquivo.

\begin{figure}[h]
  \centering
  \includegraphics[width=0.9\textwidth]{figuras/control_pacote.eps}
  \caption{Arquivo control de um pacote Debian}
  \label{fig:control_pacote}
\end{figure}

Os principais dados usados para compor o perfil do usuário são a
descrição e as \textit{Debtags} de cada pacote instalado.

As \textit{Debtags} são basicamente uma forma de melhor categorizar um pacote fora das 33
seções disponíveis no Debian, além também de permitir que um pacote seja
categorizado em diferentes frentes ao mesmo tempo \cite{zini2005cute}. Isso pode ser visto na Figura
\ref{fig:control_pacote}, onde o software \textit{vim} possui as
\textit{Debtags} \textit{devel::editor} e \textit{interface::commandline}, ou seja, está se
enquadrando tanto como um editor de texto e uma aplicação que roda em linha de
comando.

O processo de criação de \textit{Debtags} é manual, ou seja, um usuário deve associar uma
\textit{Debtag} a um pacote e a comunidade Debian deve verificar se aquela
\textit{Debtag} sugerida condiz com a aplicação em si. Atualmente, existem 45432 pacotes
categorizados com Debtags \footnote{Dados
coletados dia 06/12/2015}.

Por fim, vale lembrar que nem todas as \textit{Debtags} são consideradas significativas
para o processo de recomendação, sendo que uma filtragem é realizada visando
selecionar apenas as \textit{Debtags} válidas para a construção do perfil do usuário
\cite{araujo2011apprecommender}.

A segunda fonte de dados usada é a descrição do pacote, sendo no caso cada termo
individual. A descrição de um pacote é realizada pelo empacotador na hora com
que o mesmo cria o arquivo \textit{control}.

Com essas duas informações, o \textit{AppRecommender} cria então um perfil de usuário
onde os termos de maior peso são usados para compor tal perfil. Isso é feito por
algoritmos como \textit{Term Frequency Inverse Document Frequency Sub-Linear}
(TFIDF-sublinear).

O perfil de usuário é então uma lista contendo os termos e \textit{Debtags} de maior
peso. Este perfil é usado então para consultar toda a base de pacotes
do usuário, baseado em seu \textit{sources.list}.
Isso é realizado através do \textit{apt-xapian-index}, uma aplicação
que provê uma forma de buscar pacotes no Debian por meio de \textit{queries}, sendo que no
caso do \textit{AppRecommender}, a \textit{query} de busca é o perfil gerado para o usuário. Dessa
forma, a \textit{query} gerada é composta por cada item do perfil do usuário, sendo que a
aplicação retorna os resultados que apresentam o maior número de termos passados
na \textit{query}.
Vale ressaltar que este pacote é usado para construir todos os índices de busca
apontados pela Tabela \ref{tab:dados_apprecommender}.
Finalmente, os pacotes retornados são a recomendação final da aplicação.

Com isso dito, o \textit{AppRecommender} implementa estratégias de recomendação que
misturam algum dos dados na geração do perfil do usuário. Existem estratégias
que usam apenas \textit{Debtags}, enquanto outras combinam de forma balanceada termos da
descrição com \textit{Debtags} no perfil do usuário gerado.


\subsection{Recomendação colaborativa}

O principal dado usado na recomendação colaborativa são as submissões do
\textit{popularity-contest}. O \textit{popularity-contest} é um sistema usado pelo Debian para
acompanhar o uso dos pacotes sendo mantidos pelo sistema e ajudar também na
escolha dos pacotes que devem ser selecionados para o CD de instalação do
Debian, onde os mais populares são escolhidos \cite{araujo2011apprecommender}.

No momento que o usuário está instalando o sistema Debian, o mesmo é
questionado se gostaria de participar do
\textit{popularity-contest}. Caso o usuário diga sim, o mesmo irá enviar periodicamente
uma submissão ao \textit{popularity-contest} contendo informações de todos os seus
pacotes e o tempo de acesso e modificação de acada um deles. A Figura
\ref{fig:submissao_popcon} mostra uma submissão típica.

\begin{figure}[h]
  \centering
  \includegraphics[width=0.9\textwidth]{figuras/submissao_popcon.eps}
  \caption{Arquivo de submissão para o popularity-contest}
  \label{fig:submissao_popcon}
\end{figure}

Conforme pode ser observado na Figura \ref{fig:submissao_popcon}, após a
primeira linha de cabeçalho, cada linha da submissão possui o seguinte formato:

\begin{table}[h!]
\centering
\newcommand\T{\rule{0pt}{2.8ex}}
\newcommand\B{\rule[-1.8ex]{0pt}{0pt}}
\begin{tabularx}{15cm}{| c | X |}
\hline
\rowcolor[HTML]{EFEFEF}
{\textbf{Campo}} & \textbf{Descrição} \\ \hline
{Tempo de acesso}  & Última vez que o arquivo foi acessado \\ \hline
{Tempo de mudança}   & Última vez que informações do \textit{inode} do arquivo foi
                       modificada\\ \hline
{Nome}   & Nome do pacote                  \\ \hline
{Caminho do pacote}   & Local onde o pacote se encontra no sistema \\ \hline
{Tag}   & Campo opcional que pode classificar o pacote com algumas considerações
temporais, como se o mesmo é recente ou antigo\\ \hline
\end{tabularx}
\caption{Descrição dos campos de uma submissão ao popularity-contest}
\label{tab:submissao_popcon}
\end{table}

Baseado nestas informações, o perfil do usuário em uma recomendação colaborativa
é criado baseado em uma submissão \textit{popularity-contest}, extraindo do mesmo os pacotes usados e
as informações temporais do mesmo.

Para a recomendação propriamente dita, os vizinhos mais próximos do usuário são
aqueles com submissões mais próximas do perfil do usuário. Sendo assim, para as
recomendações colaborativas, é necessário conter as submissões de vários
usuários do \textit{popularity-contest}, para que a vizinhança criada seja significativa.

\subsection{Contexto temporal} \label{sec:contexto_temporal}

Dado esse conjunto de dados usados no \textit{AppRecommender}, a pesquisa resolveu
adicionar o contexto temporal apenas em recomendações baseadas em conteúdo. Isso
se dá pelo fato das recomendações colaborativas já usarem tal atributo e,
principalmente, pela dificuldade na obtenção dos dados necessários para execução
das estratégias colaborativas.

Com isso dito, o novo atributo que esta pesquisa pretende
adicionar é o contexto de uso de um pacote, ou seja, a última vez que o
mesmo foi usado. Em distribuições Debian, isso pode ser verificado via comando
\textit{stat}, que recebe como parâmetro o caminho para um pacote.
A saída deste programa pode ser visualizada na Figura \ref{fig:comando_stat}.

\begin{figure}[h]
  \centering
  \includegraphics[width=0.9\textwidth]{figuras/comando_stat.eps}
  \caption{Saída do comando \textit{stat} para um dado software}
  \label{fig:comando_stat}
\end{figure}

Pode ser visto na Figura \ref{fig:comando_stat} que o comando exibe os seguintes
atributos de tempo de um dado arquivo \cite{1_haas}:

\begin{itemize}
    \item \textbf{Access:} Última vez que um arquivo foi acessado, ou seja, a
        última vez que seu conteúdo foi de fato acessado.
    \item \textbf{Modify:} Última vez que conteúdo de um arquivo foi modificado.
    \item \textbf{Change:} Última vez que o status de um arquivo foi modificado,
        como mudar as permissões do arquivo ou seu dono, ou seja, modificar o
        \textit{inode} do arquivo.
\end{itemize}

Dessa forma, pode ser visto que o campo \textit{Access} de um arquivo é um dos
principais indicadores de uso de um pacote. Entretanto, o sistema de arquivos
usado pode apresentar uma
restrição neste campo. Isso acontece devido a como o sistema de arquivos é
configurado, pois algumas flags especiais podem ser usadas para alterar como
esse campo é preenchido \footnote{https://wiki.debian.org/fstab}:

\begin{itemize}
    \item \textbf{noatime:} O sistema de arquivos não irá computar o tempo de
        acesso do arquivo.
    \item \textbf{stricatime:} O sistema de arquivos irá sempre alterar o tempo
        de acesso do arquivo.
    \item \textbf{relatime:} O sistema de arquivos só irá computar o tempo de
        acesso de um arquivo se o valor desse campo for menor que os valores dos
        campos \textit{Modify} e \textit{Change}. Esse opção é default desde
        o kernel 2.6.30. Entretanto, novas adições foram feitas, permitindo que
        o tempo de acesso seja modificado se o mesmo for mais antigo que um dia.
\end{itemize}

O uso do relatime passou a ser default para montar o sistema de arquivos devido
aos problemas de perfomance ocasionados pelo atime. Segundo Ingo Molnar, "atualizar o valor
de atime é de longe uma das maiores deficiências de performance que o Linux possui
hoje" \cite{3_corbet_2007}. Sendo assim, as informações de contexto que serão
usadas, tempo de acesso, modificação e mudança, são aproximações e não
refletem perfeitamente o contexto de uso dos pacotes.

Com isso dito, a informação temporal de cada pacote será obtida pela seguinte
fórmula:

PesoTemporalDoPacote = $\frac{TempoAcesso - TempoModificação}{TempoAtual -
TempoModificação}$


Onde:

\begin{itemize}
    \item \textbf{TempoAcesso:} Tempo de último acesso do pacote em segundos.
    \item \textbf{TempoModificação:} Tempo de última modificação do pacote em
        segundos.
    \item \textbf{TempoAtual:} Tempo atual no qual o usuário executa a
        aplicação em segundos.
\end{itemize}


A equação apresentada irá retornar um valor de 0 a 1, sendo que quanto mais
perto de 1, mais recente é considerado o arquivo. Vale ressaltar que a fórmula
também leva em conta o tempo de modificação, pois em algumas situações alterar o
tempo de modificação também altera o tempo de acesso, podendo ocasionar em um
pacote recentemente atualizado, mas não usado, com um peso maior do que um
pacote recentemente usado.

Vale ressaltar que existe uma limitação quanto à essa abordagem, o caso de
pacotes que não são diretamente executados pelo usuário. Alguns pacotes são
executados toda vez que um usuário realiza certa tarefa, ou até mesmo quando o
sistema é iniciado. Dessa forma, a fim de remover alguns pacotes, decidiu-se
remover da lista de pacote dos usuários os pacotes com algumas prioridades definidas.
A prioridade de um pacote pode ser vista também no arquivo \textit{control} de um pacote, conforme visto
na Figura \ref{fig:control_pacote}. O Debian permite que cada pacote tenha
prioridade classificada mediante os seguintes valores:

\begin{itemize}
    \item \textbf{Required:} Pacotes necessários para o funcionamento básico do
        sistema
    \item \textbf{Important:} Pacotes esperados em qualquer distribuição Unix
    \item \textbf{Standard:} Pacotes que fornecem alguns serviços básicos
        esperados, como aplicações para mandar email, como o mutt.
    \item \textbf{Optional:} Pacotes que fornecem serviços que não se encaixam
        em nenhuma das categorias mencionadas acima.
    \item \textbf{Extra:} Pacotes que conflitam com outros de hierarquia maior
        ou que possuem dependências especiais que não permitem que os mesmos
        sejam categorizados como \textit{Optional}.
\end{itemize}

Baseado nestas prioridades, resolveu-se por retirar do perfil do usuário
qualquer pacote com prioridade \textit{"Important"} \textit{"Standard"}
\textit{"Required"}. Vale ressaltar que algumas linguagens de programação
possuem uma das prioridades sendo filtradas, mas um processamento adicional é
realizado para garantir que os pacotes relacionados a linguagens de programação
continuem no perfil do usuário.

\section{Teste de hipótese}

Esta seção irá descrever como esta pesquisa pretende responder às hipóteses
levantadas, mostrando assim como as duas abordagens propostas pelas hipóteses
levantadas no projeto serão implementadas, assim como os resultados das mesmas
serão comparadas entre si e com as abordagens já usadas no projeto selecionado.

\subsection{Abordagem determinística}

A proposta da abordagem determinística é de usar uma fórmula que altere
a criação do perfil de usuário através dos pacotes instalados que foram mais
recentemente usados. A Figura \ref{fig:abordagem_deterministica}
contém o fluxo usado para realizar uma
recomendação utilizando a abordagem determinística.

\begin{figure}[h]
  \centering
  \includegraphics[width=0.9\textwidth]{figuras/abordagem_deterministica.eps}
  \caption{Processo usado para realizar a abordagem deterministica}
  \label{fig:abordagem_deterministica}
\end{figure}

A abordagem determinística é a combinação da estratégia TFIDF combinada com
uma fórmula para adicionar o peso do tempo ao peso calculado pelo TFIDF. Para
isso acontecer, um dicionário associa cada termo, \textit{Debtag} ou termo da
descrição, aos pacotes onde os mesmos se encontram. Com esse dicionário criado,
no momento que o TFIDF for calculado, o peso temporal do termo também é
calculado e seus valores são enfim multiplicados.

Para calcular o peso temporal de um dado termo, calcula-se primeiro o valor da
classificação no tempo para cada pacote através da fórmula definida na Seção
Contexto Temporal \ref{sec:contexto_temporal} para atribuir o valor
a cada um dos pacotes.

Com esse valor calculado, é necessário atribuir mais peso para pacotes que
foram recentemente usados, sendo assim é usado uma função de decaimento
exponencial para obter controle quanto à suavidade da curva que relaciona
a classificação no tempo com o peso do pacote, utilizando a seguinte função:

\textbf{PesoPacote} = C / $\exp\left(-({1 - PesoTemporalDoPacote}) * {\lambda}\right)$

Onde C e alfa são constantes usadas para alterar a velocidade do decaimento da
curva e o peso dado a variável \textit{PesoTemporalDoPacote}, respectivamente.

Após calcular o peso de cada pacote do termo, o valor temporal de um termo pode ser
descrito através das seguintes funções matemáticas, onde cada termo ``T'', possui uma
lista de pacotes ``P'', onde essa lista possui um tamanho ``T''.

$T = \{P_1, P_2, ..., P_N\}$
\\

Além disso, cada pacote ``N'' possui um peso ``w'' com valores entre 0 e 1. Esse peso
é definido pela função W(P), que foi previamente explicada na Seção
\ref{sec:contexto_temporal}

$W(Pi) = w_i , \quad 0 <= w_i <= 1$
\\

Aplicando essa função para cada pacote do termo é formada a lista dos pesos
dos pacotes. Essa lista é então ordenada, gerando a lista
dos pesos ordenados theta ``$\theta$''.

$sort(\{w_1, w_2, ..., w_n\}) = \{\theta_1, \theta_2, ..., \theta_n\}$
\\

Para calcular o peso do termo total, são utilizados os cinco maiores pesos de pacote
associados ao mesmo, ou seja, se algum termo possui uma lista com tamanho ``N''
menor do que cinco pacotes, onde ``N'' indica o tamanho da lista, para cada peso
necessário para completar o valor total de cinco um pacotes, um novo peso
é definido como igual ao peso anterior com um decréscimo de $\alpha$, onde
através de experimentações, o valor de $\alpha$ foi definido como 0.05.

$\theta_j = \theta_{j-1} - \alpha , \quad j = n+1, n+2, ..., 4$
\\

Por fim, o valor temporal do termo é calculado pela média aritmética desses
cinco pesos de pacote.

$ValorTemporalDoTermo = \frac{1}{5} * \sum\limits_{i=1}^{5} \theta_i$

Com o valor temporal calculado, pode-se enfim multiplicar tal valor com o TFIDF
obtido para o dado termo. Este processo é realizado para todos os termos usados
para se obter o perfil do usuário.

\subsection{Aprendizado de máquina}

Um dos problemas de se usar apenas a abordagem determinística é uma limitação
encontrada no \textit{apt-xapian-index}, onde não é possível priorizar termos
dentro de uma \textit{query} quando se faz uma busca. Por exemplo, dado que o perfil
do usuário possui os termos ``python'' e ``ruby'', sendo que o primeiro termo
apresentou um peso maior que o segundo, na hora da criação da \textit{query}, os pesos
dos termos são ignorados. Logo, quando a busca for realizada os termos serão
equivalentes, mesmo que para o perfil do usuário eles não o sejam.

Dessa forma, para a abordagem por aprendizado de máquina, escolheu-se aplicar a
metodologia de filtragem pós-recomendação. Para o projeto em mãos, isso
significa que o mesmo irá continuar fazendo a recomendação conforme foi
desenhado, mas as recomendações serão filtradas pelo perfil de usuário
construído pela abordagem de aprendizado de máquina.

Sendo assim, o enfoque dessa abordagem é usar um algoritmo supervisionado
para criar o perfil do usuário baseado em seus pacotes. Dessa forma, é
necessário definir como cada pacote será apresentado para algoritmo, além de
definir também a categoria de cada pacote.

Para a categoria do pacote, definiu-se a seguinte escala de classificação baseada na
fórmula proposta na Seção Contexto Temporal \ref{sec:contexto_temporal}:

\begin{table}[h]
\centering
\begin{tabular}{cc}
\hline
\rowcolor[HTML]{EFEFEF}
{Escala} & {Valores} \\ \hline
{ReallyUseful(RU)}  & PesoTemporalDoPacote >= 0.8                  \\ \hline
{Usefull(U)}   & PesoTemporalDoPacote >= 0.7                  \\ \hline
{NotUsefull(NU)}   & PesoTemporalDoPacote >= 0.5                  \\ \hline
\end{tabular}
\caption{Escala para classificação de um pacote baseado em seus atributos de tempo}
\label{tab:classificacao_pacotes}
\end{table}


Com a escala definida e as informações de tempo coletadas para cada pacote,
pode-se então criar a classificação de cada pacote manualmente instalado pelo
usuário.
Isso é realizado mediante um script automatizado \footnote{\url{https://github.com/TCC-AppRecommender/AppRecommender/tree/master/src/bin}}
que irá realizar tal função.

Uma vez classificados, os pacotes então podem ser usados como entrada para o
algoritmo supervisionado, visando assim gerar o perfil do usuário.
Entretanto, ainda é necessário definir como os dados presentes no pacote serão usados de entrada para o
algoritmo de aprendizado supervisionado. Considerando que os principais dados do pacote sendo usados na
recomendação são as \textit{Debtags} e os termos que compõem sua descrição, escolheu-se usar uma abordagem de
vetor binário para cada um dos items. Sendo assim, o vetor de entrada será composto pela agregação de dois vetores
binários.

Para as \textit{Debtags}, o vetor binário irá ter o tamanho igual ao número de
\textit{Debtags} válidas definidas pelo AppRecommeder, sendo no caso 276
\textit{Debtags} \cite{araujo2011apprecommender}. Dessa forma,
o vetor binário de um dado pacote irá conter o valor ``0'' caso o mesmo não
contenha uma Debtag e ``1'' caso a mesma se encontre
no pacote. Por exemplo, dado um conjunto de três Debtags, \textit{role:program},
\textit{implemented-in:c} e \textit{devel:editor}, e um pacote com a Debtag
\textit{role:program} associado a ele, o seu vetor binário seria o seguinte: [0, 0, 1].

Para os termos da descrição, a formatação do vetor é um pouco diferente. Primeiramente, optou-se por usar os termos apenas presentes
nos pacotes manualmente instalados do usuário, e não na lista de pacotes válidos
usados pela aplicação. Isso se dá para reduzir o tamanho do vetor e também
pelo fato de que se um termo não aparece em nenhum pacote instalado pelo
usuário, o mesmo, em tese, não terá nenhum peso relevante na
classificação. Além disso, usou-se a estratégia de TFIDF para retirar termos não
relevantes. Para isso, calculou-se o peso TFIDF de todos os termos do usuário.
Com essa informação em mãos, a média do TFIDF é encontrada e termos com valores menores que a
média não são usados no vetor binário. Após aplicação deste filtro, a abordagem de vetor binário pode ser usada, funcionando de forma
similar ao vetor binário das Debtags, onde os termos estarão representados em ordem alfabética e caso o pacote contenha tal termo em sua
descrição, a posição do vetor associado ao termo será marcada como ``1''.

Com a combinação desses valores, chega-se então à forma final da entrada de um pacote para o uso de algoritmos de aprendizado
supervisionados. Observa-se também que para propósitos de armazenamento, a classificação do pacote também é armazenada ao final da combinação
dos vetores binários dos termos e Debtags. O papel desse vetor pode ser visto na
Seção \ref{ape:bayes_ingenuo}, onde tal vetor binário representa uma linha da
matrix ``D'' e a classificação representa uma linha da matrix ``L''. Logo, o
conjunto de todos os pacotes que serão usados no algoritmo formam de fato a
matrix ``D''.

Por fim, os pacotes então podem finalmente ser usados para alimentar o algoritmo de aprendizado escolhido para a pesquisa, sendo no caso um
classificador bayesiano. A escolha desse classe de algoritmo se deu por alguns fatores. O primeiro se dá pela popularidade do algoritmo para
sistemas de recomendação por conteúdo \cite{amatriain2011data}. Outro fator que levou a escolha desse algoritmo é a sua capacidade de produzir
modelos eficazes sem necessitar um enorme processamento de dados, além do fato
de que o modelo gerado é facilmente compreensível, diferente de outros métodos,
como redes neurais\cite{segaran2007programming}.

Vale ressaltar também que algoritmos de aprendizado supervisionados não-lineares foram considerados, como redes neurais. Entretanto, segundo
a pesquisa realizada por \cite{pazzani1997learning} o uso de redes neurais em relação ao classificador bayesiano não mostra nenhuma melhoria
considerável para sistemas de recomendação baseado em conteúdo. Sendo assim, optou-se por usar o
classificador bayesiano de início e caso o modelo não apresente
resultados satisfatórios, mudar então para um algoritmo supervisionado não-linear.

É importante também ressaltar que o algoritmo será validado não só pela técnicas
usadas no \textit{AppRecommender}, mas também técnicas próprias de
aprendizado de máquina serão usadas para verificar se o modelo gerado está
satisfatório, como a validação cruzada, sendo o foco principal a
identificação de \textit{overfitting} e \textit{underfitting}.

Com o perfil de usuário criado e validado, toda vez que uma recomendação for gerada, os pacotes escolhidos serão filtrados pelo perfil criado,
sendo que apenas os pacotes classificados como ``\textit{RU}'' serão qualificados para a recomendação ao usuário.

\section{Comparação dos resultados}

Esta seção é destinada a demonstrar como as soluções implementadas serão
comparadas entre si e entre as soluções já implementadas pelo projeto
selecionado. Vale ressaltar que a comparação de resultados visa principalmente
responder as hipóteses levantadas na pesquisa.

Com isso em consideração, esta pesquisa escolheu seguir o modelo proposto na
implementação do \textit{AppRecommender}, entretanto com o foco nas
coletas de opinião com usuários. Isso se dá pelo aspecto mais subjetivo das
hipóteses da pesquisa.

Além disso, essa pesquisa tem como foco realizar duas coletas de opinião com usuários,
sendo a primeira coleta podendo ser considerada um teste piloto, onde essa
coleta será realizada com um grupo de usuários, afim de encontrar problemas e
observações para melhorar as soluções propostas. A segunda coleta de opinião
terá foco então em propor alternativas para as questões levantadas
primeira coleta, visando assim tentar melhorar seus resultados.

\subsection{Estudo com usuários}\label{sec:estudo_usuario}

Devido a subjetividade inerente sobre o que um usuário considera uma boa
recomendação, esta pesquisa irá focar em estudos de usuários para
avaliar as soluções implementadas. Para essa estratégia, será seguido um modelo
similar ao usado nos testes de usuário do próprio \textit{AppRecommender}. Entrentato
algumas modificações serão realizadas no processo.

O teste de usuário realizado pelo \textit{AppRecommender} foi realizado online, onde cada
usuário tinha acesso a uma aplicação rodando o \textit{AppRecommender}. Essa aplicação
funcionava como front-end para a aplicação e escolhia uma estratégia aleatória
do \textit{AppRecommender} para ser avaliada. O usuário então teria acesso a 10
recomendações, onde cada recomendação provê uma descrição do
pacote, o link para seu \textit{upstream} e uma imagem do pacote em uso. O
usuário então avaliaria o pacote recomendado e seguiria para o próximo pacote.
Ao final de uma recomendação, o usuário poderia então escolher se gostaria de
analisar uma outra estratégia de recomendação, onde então o processo se
repetiria.

Como modificações neste processo, esta pesquisa resolveu realizar o estudo de
usuário de forma offline. Além disso, o processo de teste também foi alterado.
Para esta pesquisa, resolveu-se comparar uma estratégia já implementada pelo
\textit{AppRecommender} com as outras estratégias implementadas nesta pesquisa.
Além disso, resolveu-se também usar uma abordagem por linha de
comando, com o intuinto de agilizar os testes de usuário e possibilitar que os
mesmos possam ser repetidos mais vezes.

Dessa forma, o teste de usuário contará com a recomendação de ``x'' pacotes por
estratégia usada. Tal valor será decidido de acordo com o número de estratégias
usadas. A ordem com que
as recomendações são realizadas é aleatória e as informações apresentadas
para cada pacote serão as mesmas definidas pelo próprio \textit{AppRecommender}, com
exceção da foto do pacote. Para cada pacote recomendado, o usuário irá poder
avaliar o mesmo pelo linha de comando. Após avaliar todos os pacotes, o teste
será finalizado e as informações serão coletadas pelos responsáveis pela
pesquisa.

Para as métricas e a escala de avaliação, resolveu-se usar as mesmas definidas
pelo AppRecommender \cite{araujo2011apprecommender}. A escala de avaliação
possui quatro itens que são definidos a seguir:

\begin{itemize}
    \item \textbf{Ruim: } Recomendação que não agrada ao usuário.
    \item \textbf{Redundante: } Usuário possui aplicativos similares para o item
        sendo recomendado.
    \item \textbf{Útil: } Usuário acha que a recomendação lhe proporciona um
            pacote útil.
    \item \textbf{Surpresa boa: } Usuário considera a recomendação útil e além
        do mais inesperada.
\end{itemize}

Para as métricas usadas considera-se que as avaliações Ruim e Redundante são
verdadeiro negativos e Útil e Surpresa boa são verdadeiro positivos. Além disso,
quando se avalia na linha de comando, as mesmas serão numeradas de 1 até 4.

Considerando isso, as métricas usadas para comparar as estratégias são as
seguintes:

\begin{itemize}
    \item \textbf{Precisão: } $\frac{NumUtil + NumSurpresaBoa}{NumRecomendacoes}$
    \item \textbf{Novidade: } $\frac{NumSurpresaBoa}{NumRecomendacoes}$
    \item \textbf{Ruim: } $\frac{NumRuim}{NumRecomendacoes}$
\end{itemize}

Onde:

\begin{itemize}
    \item \textbf{NumUtil:} Número de recomendações marcadas como \textit{Útil}
        pelo usuário.
    \item \textbf{NumSurpresaBoa:} Número de recomendações marcadas como
        \textit{Surpresa boa} pelo usuário.
    \item \textbf{NumRecomendacoes:} Total de recomendações geradas para o
        usuário.
    \item \textbf{NumRuim:} Número de recomendações marcadas como \textit{Ruim}
        pelo usuário.
\end{itemize}

Sendo assim, cada recomendação usada será avaliada por essas três métricas no
contexto de um usuário. Ao final, as médias dessas métricas serão calculadas para
todos os usuários participante da pesquisa, com o intuito de comparar as
estratégias de maneira siginificativa.
