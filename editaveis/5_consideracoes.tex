\chapter{CONSIDERAÇÕES PRELIMINARES}
\label{chapter:final}

As contribuições deste trabalho propiciaram ao InterSCity a possibilidade de
atuação em cenários mais extremos e abrangentes, por meio de um novo serviço
para processamento de seus dados. Esse novo serviço deu origem a uma aplicação
que abstrai as ferramentas de Big Data utilizadas pelo projeto, e que fornece
meios para que outras aplicações de cidades inteligentes requisitem tarefas para
serem processadas pela plataforma.

Esta primeira etapa do trabalho envolveu a revisão bibliográfica e técnica
sobre arquiteturas e tecnologias de Big Data, assim como a concepção do serviço
de processamento de dados para o InterSCity. Portanto, ainda existem várias
melhorias a serem feitas na implementação inicial do Shock, como a resolução de
dívidas técnicas e apuração de maior referencial científico. As pendências
presentes no novo serviço de processamento estão planejadas para serem
resolvidas conforme o cronograma apresentado na Tabela \ref{tab:cronograma}.

\begin{table}[h]
  \begin{center}
    \begin{tabular}{|l| p{12cm}|}
        \hline \textbf{Mês} & \textbf{Atividade} \\

        \hline Maio & Revisão mais profunda na bibliografia a respeito das
        tecnologias e arquiteturas no contexto de cidades inteligentes.\\

        \hline Maio & Desacoplar o Kafka do \textit{core} do Shock. \\

        \hline Maio & Testar o \textit{core} do Shock. \\

        \hline Junho & Documentar a API de serviços. \\

        \hline Junho & Utilizar Data Frames e outras estruturas de dados do
        Spark, e não só RDD's. \\

        \hline Junho & Disponibilizar customização de janelas de
        \textit{micro-batch}. \\

        \hline Julho & Permitir múltiplos \textit{streams} através da mesma
        API (possibilitando o uso do Kinesis, HDFS, Sockets, entre outros). \\

        \hline Julho & Adicionar o controle de \textit{check-points}. \\

        \hline Julho & Utilizar recuperação de dados históricos através dos
        \textit{logs} do Kafka. \\

        \hline Agosto & Agregar o novo serviço de processamento e as mudanças
        necessárias para o núcleo do InterSCity. \\

      \hline
    \end{tabular}
  \end{center}
  \caption{Cronograma com o planejamento das dívidas técnicas.}
  \label{tab:cronograma}
\end{table}
