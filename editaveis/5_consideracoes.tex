\chapter{CONSIDERAÇÕES INICIAIS}
\label{chapter:final}

As contribuições deste trabalho propiciaram ao InterSCity a possibilidade de
atuação em cenários mais extremos e abrangentes, por meio de um novo serviço
para processamento de seus dados. Este novo serviço deu origem a uma aplicação
que abstrai as ferramentas de Big Data utilizadas pelo projeto, e que fornece
meios para que outras aplicações de cidades inteligentes requisitem tarefas para
serem processadas pela plataforma. Para ilustrar o uso do novo serviço, foi
desenvolvido uma aplicação teste que utiliza um dos exemplos desenvolvidos
pelo time do InterSCity, e que registra no Shock operações para serem feitas,
apresentando ao final os resultados do processamento.

O Shock, contudo, não se encontra em um estado ideal, e dívidas técnicas
encontram-se presentes. O núcleo da aplicação encontra-se acoplado ao Kafka e
ao Spark, dificultando o desenvolvimento dos testes, não presentes neste
momento. A documentação do Shock é frágil e quase inexistente, o que atrapalha
a evolução do Shock por outros membros do InterSCity. O Shock em seu
desenvolvimento utiliza somente RDD's como estrutura de dados, ignorando outras
estruturas importantes e mais rebuscadas, como os Data Frames. Outro aspecto
que não houve grande cuidado foi a decisão a respeito das janelas de execução
dos \textit{micro-batches}, que utilizam o mesmo valor, mas que podem ser
customizadas pelo Spark, e que fornece também o uso de \textit{check-points}
para serem utilizados caso erros graves ocorram, também não utilizados  pelo
Shock. Por fim, o Shock neste momento não recupera informações do MongoDB,
não possibilitando assim a análise de dados históricos.

É esperado que na continuidade do projeto sejam sanadas as dívidas técnicas
relatadas, e, ao final, que o Shock seja analisado pelo time do InterSCity,
sendo adotado pela plataforma, compondo assim o núcleo do projeto.
