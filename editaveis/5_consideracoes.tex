\chapter{CONSIDERAÇÕES PRELIMINARES}
\label{chapter:final}

As contribuições deste trabalho propiciaram ao InterSCity a possibilidade de
atuação em cenários mais extremos e abrangentes, por meio de um novo serviço
para processamento de seus dados. Esse novo serviço deu origem a uma aplicação
que abstrai as ferramentas de Big Data utilizadas pelo projeto, e que fornece
meios para que outras aplicações de cidades inteligentes requisitem tarefas para
serem processadas pela plataforma. Com a incorporação do novo serviço de
processamento o InterSCity passa então: a poder atuar em operações com grande
massa de dados; a fornecer processamento de dados para terceiros através da
plataforma; e a possibilitar o uso de algorítmos mais complexos, como de
aprendizado de máquina.

Esta primeira etapa do trabalho começou em março, e envolveu a revisão
bibliográfica e técnica sobre arquiteturas e tecnologias de Big Data, assim
como a concepção do serviço de processamento de dados para o InterSCity.
Portanto, ainda existem várias melhorias a serem feitas na implementação
inicial do Shock, como a resolução de dívidas técnicas e apuração de maior
referencial científico. A continuação do projeto focará então nas seguintes
atividades:

\begin{itemize}
    \item \textbf{Revisão mais profunda na bibliografia}: a revisão feita não
        foi profunda o suficiente, e foi difícil pelo fato das tecnologias e
        técnicas utilizadas serem recentes. Uma maior revisão bibliográfica
        trará mais insumos para justificativas e adaptações a serem
        implantadas no InterSCity;

    \item \textbf{Desacoplar o Kafka do núcleo do Shock}: o Kafka neste
        momento está acoplado ao Shock em diversos aspectos, de modo que a
        escrita de testes seja difícil, e impossibilitando o uso de outros
        \textit{streams} no Spark, como o de MQTT;

    \item \textbf{Testar o núcleo do Shock}: por conta do pouco tempo e do
        acoplamento entre o Kafka e o Shock, os testes não puderam ser feitos;

    \item \textbf{Documentar a API de serviços}: por conta da alta volatilidade
        da arquitetura e da API do Shock, uma melhor documentação não pôde ser
        produzida. Uma melhor documentação ajudará o time do InterSCity a
        prosseguir com o serviço desenvolvido;

    \item \textbf{Utilizar outras estruturas de dados}: atualmente o Shock só
        conta com RDD's, que são estruturas elementares do Spark. A troca por
        outras estruturas, como os Data Frames, pode melhorar a performance e a
        manutenibilidade do serviço;

    \item \textbf{Customização de janelas de \textit{micro-batches}}: neste
        momento o Shock sempre utiliza o mesmo tempo de janela de
        \textit{micro-batch}. Uma customização ajudaria na adaptação do Shock
        em um maior número de contextos, sendo útil inclusive no
        desenvolvimento de testes;

    \item \textbf{Múltiplos \textit{streams}}: o Shock só utiliza os
        \textit{streams} do Kafka, e sempre um \textit{stream} único. Com
        múltiplos \textit{streams} seria possível ter múltiplos \textit{
        pipeline de dados}, interessante para as aplicações que utilizam
        a plataforma;

    \item \textbf{Controle de \textit{check-point}}: o controle de
        \textit{check-points} adicionará maior tolerância ao Shock em caso
        de falhas graves, diminuindo os pontos de ruptura;

    \item \textbf{Utilizar \textit{log} do Kafka}: a utilização do \textit{log}
        do Kafka possibilitará a recuperação de dados históricos;

    \item \textbf{Introduzir o Shock ao \textit{core} do InterSCity}: o Shock
        não está em um estado estável o suficiente para fazer parte do
        InterSCity. Quando o Shock estiver nesse nível de estabilidade, ele
        poderá finalmente fazer parte do núcleo da plataforma.
\end{itemize}

As pendências
presentes no novo serviço de processamento estão planejadas para serem
resolvidas conforme o cronograma apresentado na Tabela \ref{tab:cronograma}.

\begin{table}[h]
  \begin{center}
  \caption{Cronograma com o planejamento das dívidas técnicas.}
    \begin{tabular}{|l| p{12cm}|c|}
        \hline \textbf{Mês} & \textbf{Atividade}  & \textbf{Esforço} \\

        \hline Maio & Revisão mais profunda na bibliografia a respeito das
        tecnologias e arquiteturas no contexto de cidades inteligentes. &
        Alto \\

        \hline Maio & Desacoplar o Kafka do núcleo do Shock. & Médio \\

        \hline Maio e Junho & Testar o núcleo do Shock. & Alto \\

        \hline Maio e Junho & Documentar a API de serviços. & Baixo \\

        \hline Junho & Utilizar Data Frames e outras estruturas de dados do
        Spark, e não só RDD's. & Médio \\

        \hline Junho & Disponibilizar customização de janelas de \textit{micro-batch}. & Baixo \\

        \hline Julho & Permitir múltiplos \textit{streams} através da mesma
        API (possibilitando o uso do Kinesis, HDFS, Sockets, entre outros).
          & Alto \\

        \hline Julho & Adicionar o controle de \textit{check-points}. 
            & Médio \\

        \hline Julho & Utilizar recuperação de dados históricos através dos
        \textit{logs} do Kafka. & Médio \\

        \hline Agosto & Agregar o novo serviço de processamento e as mudanças
        necessárias para o núcleo do InterSCity. & Médio \\

      \hline
    \end{tabular}
  \end{center}
  \label{tab:cronograma}
\end{table}
