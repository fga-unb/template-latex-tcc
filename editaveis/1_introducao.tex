\chapter[INTRODUÇÃO]{INTRODUÇÃO}
\label{chapter:intro}

O termo \textbf{cidades inteligentes} recebe cada vez mais atenção, e trata-se
da utilização de tecnologias da informação e comunicação (TIC) para melhorar
setores como segurança, transporte e saúde, aumentando a qualidade de vida
da população \cite{batty2012smart}. As cidades inteligentes ganham força por
atingirem soluções e mitigações concretas para problemas graves e recorrentes
das cidades atuais, como o mau uso de recursos, a burocracia, transporte de má
qualidade e falta de segurança \cite{batty2012smart}. Para ajudar no
desenvolvimento de aplicações de cidades inteligentes são desenvolvidas
plataformas que oferecem diversos requisitos funcionais e não-funcionais,
facilitando assim o desenvolvimento de novas soluções \cite{kon2016}.

Diversas iniciativas de cidades inteligentes ocorrem atualmente. Em Santander,
na Espanha, foi desenvolvida a plataforma
SmartSantander\footnote{\url{www.smartsantander.eu/}}, e utilizando-a,
diversos aplicativos foram criados, como para apresentar informações diversas
da cidade (sobre tráfego, temperatura, transporte público), ou para informar
lugares livres para
estacionar\footnote{\url{www.smartsantander.eu/wiki/index.php/Mitos/Mitos}}
\cite{gutierrez2013}.  Em Amsterdã, na Holanda, a plataforma Amsterdam Smart
City\footnote{\url{https://amsterdamsmartcity.com}} disponibiliza serviços para
aplicações de cidades inteligentes. Apesar de existirem soluções
e propostas, vários desafios técnicos em plataformas de cidades inteligentes
ainda persistem. As soluções atuais costumam ser específicas, não promovendo
interoperabilidade entre as ferramentas e não promovendo reuso dos projetos já
desenvolvidos em outros contextos \cite{delesposte2017}.

Com a finalidade de ser uma plataforma que em sua origem se atente aos
problemas de interoperabilidade citados, surge o InterSCity, que visa
suportar o desenvolvimento de novas aplicações, projetos e serviços em cidades
inteligentes. A arquitetura da plataforma é baseada em microsserviços, e
tem como foco a interoperabilidade, a padronização, a escalabilidade e
a extensibilidade. O InterSCity está em desenvolvimento, e embora já tenha uma
arquitetura bem definida e conte com diversas funcionalidades, ainda não dispõe
de um serviço de processamento de dados adequado para contextos de larga
escala, comum em cenários de cidades inteligentes \cite{alnuaimi2015}.

Este trabalho tem como principal contribuição o planejamento, desenho e
implementação de um novo serviço de processamento de dados para o InterSCity,
permitindo assim seu uso em cenários de grande massa de dados, e possibilitando
a criação de um \textit{pipeline de dados} sofisticado e customizável. Dessa
forma, desenvolvemos o Shock, uma aplicação que abstrai o uso de tecnologias
de \textit{Big Data}, e que foi construído utilizando a Arquitetura Kappa. É
esperado também que o estudo que desenvolvemos e as implementações feitas
sejam úteis para futuras plataformas de cidades inteligentes.

Apresentamos no Capítulo \ref{chapter:interscity} maiores detalhes sobre as
características e o estado atual do InterSCity, trazendo ainda a abordagem que
utilizamos para definir o novo serviço de processamento. No Capítulo
\ref{chapter:data} trazemos um estudo sobre o estado da arte em arquiteturas e
tecnologias de \textit{Big Data}, bem como uma análise de diferentes ferramentas
adequadas para o contexto da plataforma. No Capítulo \ref{chapter:architecture}
levantamos as decisões, justificativas e resultados atingidos quanto ao novo
serviço de processamento, e por fim, no Capítulo \ref{chapter:final},
finalizamos o trabalho trazendo as considerações preliminares e os próximos
passos que tomaremos.
