\begin{resumo}[Abstract]
 \begin{otherlanguage*}{english}

   Recommender systems are becoming more popular everyday. Different
   applications which have a huge number of items or services to provide
   to their user are now using recommender systems to help users
   to find their preferences easier on the application, such as Netflix and
   Amazon. An example of applications that could benefit from recommender
   systems are GNU/Linux distributions, such as Debian. The number of
   packages for Debian systems is increasing
   everyday, with more than 49000 packages
   available to users. In that context, recommend the right package
   to the right user may not only help the user on
   performing his daily task, but also improve the community as a whole.
   This research aims to use a temporal context of the package use as a
   differential when creating the user recommendation profile, with
   the hypothesis that it will provide better recommendations. This
   will be done based on the software \textit{AppRecommender}, which
   already provides some package recommendation techniques to
   Debian users.

   \vspace{\onelineskip}
 
   \noindent 
   \textbf{Key-words}: Recommender systems, Debian, Machine learning
 \end{otherlanguage*}
\end{resumo}
