\chapter[Trabalhos futuros]{Trabalhos futuros}

Dado as questões levantadas pela pesquisa, foi possível levantar alguns
possíveis trabalhos futuros:

\begin{itemize}
  \item \textbf{Uso do arquivo history do usuário para complementar contexto
  temporal:} Apesar do arquivo \textit{bash\_history} do usuário apenas conter um
  histórico de comandos usados pelo usuário, tal arquivo pode conter informações
  importantes sobre a frequência de uso de algumas aplicações pelo usuário,
  podendo complementar a análise temporal feita.
  \item \textbf{Melhorar estratégias de aprendizado de máquina:} Dado os
  problemas já citados pelas estratégias de aprendizado de máquina, pode-se
  investigar novas metodologias para lidar com os problemas citados, como poucos
  dados de treinamento e complexidade de uso dos mesmos.
  \item \textbf{\textit{Cluster} de interesse do usuário:} Essa nova estratégia seria
  para criar \textit{clusters} de gostos do usuário com base em seus pacotes e
  buscar pacotes baseado na identidade desse \textit{cluster} e não nos pacotes
  em si.
  \item \textbf{Melhorar contexto temporal:} As estratégias de contexto
  temporal ainda apresentam problemas para pacotes recentemente atualizados.
  Isso se dá pelo fato que ainda não é possível saber se um pacote foi
  recentemente utilizado e atualizado pelo sistema, pois a operação de
  atualização opera tanto sobre o \textit{ctime} quanto \textit{atime}. Dessa
  forma, pode-se buscar maneira de resolver tal problema.
  \item \textbf{Evitar redundância de recomendação de pacotes:} Pacotes
  instalados pelo usuário por outros gerenciadores de pacotes, como
  \textit{rvm} do ruby ou o \textit{pip} do python, não devem ser
  recomendados. Dessa forma, deve-se entender um mecanismo de verificar a
  existência desses pacotes e retirá-los da recomendação. Além disso, deve-se
  verificar também redundância entre pacotes dentro de uma recomendação,
  pois dois pacotes similares não devem estar presentes na mesma
  recomendação.
  \item \textbf{Uso de experimentos offline:} Experimentos offline eram
  implementados no \textit{AppRecommender} com o intuito de comparar as
  estratégias de forma mais rápida e prática, pois não era necessário
  sempre fazer testes com usuários. Para o contexto temporal usado, tais
  experimentos deveriam ser modificados e melhor explorados para se
  adequar a finalidade da pesquisa.
\end{itemize}
