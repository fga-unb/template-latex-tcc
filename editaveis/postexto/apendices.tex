\begin{apendicesenv}
%
\partapendices
%
\chapter{PRINCÍPIOS UTILIZADOS PELO INTERSCITY}
\label{appendix:principles}

O InterSCity foi desenvolvido utilizando princípios de \textit{design}, e,
assim, busca atender critérios estabelecidos. Os princípios são:

\begin{itemize}
    \item \textbf{Modularidade através de serviços}: O InterSCity se torna mais
modular através da criação de mais microserviços, que buscam ter
responsabilidades atômicas e bem definidas \cite{delesposte2017}.

    \item \textbf{Modelos e Dados Distribuídos}: O InterSCity melhora sua
escalabilidade através da distribuição dos dados e dos modelos. Com essa
prática, cada microserviço pode evoluir separadamente, por contar com seu
próprio banco de dados \cite{delesposte2017}. Contudo, esse princípio apresenta
o ponto negativo de aumentar a complexidade.

    \item \textbf{Evolução Descentralizada}: Por conta do não-acoplamento, é
possível que módulos do InterSCity evoluam e sofram manutenção
independentemente, sem afetar outros microserviços da plataforma
\cite{delesposte2017}.

    \item \textbf{Reuso de Projetos de Código Aberto}: O InterSCity preferencia % uso software livre ou codigo aberto?
projetos robustos já desenvolvidos, ao invés de desenvolver soluções do zero
\cite{delesposte2017}. Contudo, esta escolha é feita com cuidado, e somente
projetos com colaboradores e mantenedores ativos, e com documentação
apropriada, são utilizadas na plataforma \cite{delesposte2017}.

    \item \textbf{Adoção de Padrões Abertos}: O InterSCity adota padrões já
difundidos, para que seja provida maior interoperabilidade entre a plataforma
e outros projetos \cite{delesposte2017}.

    \item \textbf{Assíncrono contra Síncrono}: O InterSCity busca prover
serviços e atividades assíncronas sempre que possível, com a finalidade de
evitar que eventos blocantes ocorram. Isto é atingido principalmente através
do padrão PubSub, e de estratégias baseadas em eventos \cite{delesposte2017}.

    \item \textbf{Serviços sem Estado}: Os microserviços do InterSCity evitam,
sempre que possível, ter um estado específico \cite{delesposte2017}. Com isso,
os microserviços podem responder a qualquer requisição a qualquer momento, ao
contrário do que ocorreria caso tivessem estados específicos, pois só
conseguiriam caso certas transições ocorressem.
\end{itemize}

    \chapter{PERFORMANCE DO INTERSCITY}
    \label{appendix:performance}
Para mensurar a performance do InterSCity, dois experimentos foram feitos:
um, para avaliar a degradação da performance conforme o número de clientes
enviando dados concorrentemente aumenta; e outra análise, com mudanças na % posso usar ponto e virgula?
infraestrutura, com o objetivo de mensurar o quanto a performance melhorou
com o aumento dos recursos computacionais \cite{delesposte2017}.

\begin{figure}
  \centering
    \includegraphics[scale=0.7]{figuras/benchmark1.png}
    \caption{Degradação no tempo de resposta. Fonte: ESPOSTE, 2017.}
  \label{fig:benchmark1}
\end{figure}

No primeiro experimento, clientes ficaram em um laço de repetição enviando
dados por requisições síncronas durante 4 minutos. O experimento foi feito
utilizando 11 diferentes configurações, variando o número de clientes
concorrentes. Com os resultados, que estão apresentados na Figura
\ref{fig:benchmark1}, pôde ser percebido que: (i) o microserviço
Resource Adaptor manteve o maior uso dos recursos, sendo considerado o
maior gargalo da plataforma, juntamente com o Data Collector, que também
apresentou uso intensivo da CPU; (ii) para 50 clientes concorrentes, o
tempo de resposta médio foi de 60 milissegundos; (iii) acima de 250 clientes
concorrentes, as requisições passaram a ter latência de 1 segundo
\cite{delesposte2017}.

\begin{figure}
  \centering
    \includegraphics[scale=0.4]{figuras/benchmark2.png}
    \caption{Número de requisições por segundo com aumento de recursos. Fonte: ESPOSTE, 2017.}
  \label{fig:benchmark2}
\end{figure}

No segundo experimento, um cenário de 500 clientes concorrentes foi utilizado, 
durante 6 ciclos de 4 minutos, e com diferentes configurações de
\textit{hardware}, melhorando os recursos computacionais. Foi utilizado
balanceamento de carga igualitário (\textit{round-robin}) nos serviços
que foram identificados como os gargalos da plataforma - isto foi possível
graças aos princípios seguidos, mencionados anteriormente. Os resultados,
que estão apresentados na Figura \ref{fig:benchmark2}, permitem concluir que:
(i) só é necessário a melhora de alguns microserviços para melhor desempenho
da plataforma, até certo ponto; (ii) a plataforma consegue escalar linearmente
com a adição de recursos, o que mostra o quão escalável é o InterSCity. Além
disso, foi observado que, por conta do alto número de clientes, o RabbitMQ
passou a consumir grandes recursos, o que o trás como outro possível gargalo,
caso tenha disponível pouco recurso computacional \cite{delesposte2017}.

\end{apendicesenv}

