\chapter[Considerações sobre os Elementos Textuais]{Considerações sobre os 
Elementos Textuais}

\section{Introdução}

A regra mais rígida com respeito a Introdução é que a mesma, que é 
necessariamente parte integrante do texto, não deverá fazer agradecimentos 
a pessoas ou instituições nem comentários pessoais do autor atinentes à 
escolha ou à relevância do tema.

A Introdução obedece a critérios do Método Cientifico e a exigências 
didáticas. Na Introdução o leitor deve ser colocado dentro do espírito do 
trabalho.

Cabe mencionar que a Introdução de um trabalho pode, pelo menos em parte, 
ser escrita com grande vantagem uma vez concluído o trabalho (ou o 
Desenvolvimento e as Conclusões terem sido redigidos). Não só a pesquisa 
costuma modificar-se durante a execução, mas também, ao fim do trabalho, o 
autor tem melhor perspectiva ou visão de conjunto.

Por seu caráter didático, a Introdução deve, ao seu primeiro parágrafo, 
sugerir o mais claramente possível o que pretende o autor. Em seguida deve 
procurar situar o problema a ser examinado em relação ao desenvolvimento 
científico e técnico do momento. Assim sendo, sempre que pertinente, os 
seguintes pontos devem ser abordados: 

\begin{itemize}

	\item Contextualização ou apresentação do tema em linhas gerais de 
	forma clara e objetiva;
	\item Apresentação da justificativa e/ou relevância do tema escolhido;
	\item Apresentação da questão ou problema de pesquisa;
	\item Declaração dos objetivos, gerais e específicos do trabalho;
	\item Apresentação resumida da metodologia, e
	\item Indicação de como o trabalho estará organizado.

\end{itemize}

\section{Desenvolvimento}

O Desenvolvimento (Miolo ou Corpo do Trabalho) é subdividido em seções de 
acordo com o planejamento do autor. As seções primárias são aquelas que 
resultam da primeira divisão do texto do documento, geralmente 
correspondendo a divisão em capítulos. Seções secundárias, terciárias, 
etc., são aquelas que resultam da divisão do texto de uma seção primária, 
secundária, terciária, etc., respectivamente.

As seções primárias são numeradas consecutivamente, seguindo a série 
natural de números inteiros, a partir de 1, pela ordem de sua sucessão no 
documento.

O Desenvolvimento é a seção mais importante do trabalho, por isso exigi-se 
organização, objetividade e clareza. É conveniente dividi-lo em pelo menos 
três partes:

\begin{itemize}

	\item Referencial teórico, que corresponde a uma análise dos trabalhos 
	relevantes, encontrados na pesquisa bibliográfica sobre o assunto. 
	\item Metodologia, que é a descrição de todos os passos metodológicos 
	utilizados no trabalho. Sugere-se que se enfatize especialmente em (1) 
	População ou Sujeitos da pesquisa, (2) Materiais e equipamentos 
	utilizados e (3) Procedimentos de coleta de dados.
	\item Resultados, Discussão dos resultados e Conclusões, que é onde se 
	apresenta os dados encontrados a análise feita pelo autor à luz do 
	Referencial teórico e as Conclusões.

\end{itemize}

\section{Uso de editores de texto}

O uso de programas de edição eletrônica de textos é de livre escolha do autor. 

