\chapter[ARQUITETURA LAMBDA]{ARQUITETURA LAMBDA}

A Arquitetura Lambda é uma abordagem para processar dados em uma plataforma
que lida com uma grande massa de dados, e surge como um caminho alternativo
a outras arquiteturas mais antigas, como a incremental com \textit{sharding}
\cite{marz2015}.

De maneira geral, a Arquitetura Lambda é composta de três camadas: a camada
\textit{batch}, que constantemente processa uma grande quantidade de dados e
demora razoávelmente em seu processamento, a camada \textit{serving},
responsável por disponibilizar os resultados do processamento, e a camada
\textit{speed}, responsável por atuar enquanto a camada \textit{batch} está
ocupada processando \cite{marz2015}. Cada uma destas camadas é implementada
utilizando algoritmos e ferramentas específicas, de modo que certas ferramentas
são mais apropriadas em certos contextos.

\section{BATCH LAYER}

Responsável pelo processamento de uma grande massa de dados com alta latência,
a camada \textit{batch} é responsável por gerenciar e criar os
\textit{Master datasets}, que podem ser entendidos como lotes de informações
preprocessadas e prontas para responderem \textit{queries}. Os dados a serem
processados na camada \textit{batch} são considerados imutáveis, de modo que,
caso uma mudança seja necessária, o dado que carece alteração não sofre
transformações, permanecendo inalterado, e um novo dado com as alterações é
inserido no lote \cite{marz2015}.

\subsection{Criação do Schema}

Uma solução comum para formato de escrita de dados é o JSON, por ser simples e
bem difundido. Contudo, esse tipo de abordagem não utiliza um \textit{schema}
para descrever os dados, o que pode ser perigoso em contextos de alta
concorrência e na utilização de tecnologias Big Data. A utilização do
\textit{schema} é mais segura com relação a chance de corrupção dos dados,
graças a maior descrição do contexto dos dados \cite{marz2015}.

Sendo assim, para utilização, processamento e armazenamento de dados na
camada \textit{batch}, é interessante a utilização de um \textit{framework}
de serialização; com eles, são gerados dados que depois podem ser lidos
e escritos na tecnologia que o projeto desejar, desde que estejam de acordo
com o \textit{schema} estabelecido.

O processo de serialização de dados, contudo, implica em algumas restrições.
Os dados serializados são imutáveis, sendo então restrito suas alterações ou
deleções diretamente. Quando alguma mudança precisar ser feita, novos dados
deverão ser inseridos, e essas novas adições contemplarão as alterações 
e deleções.
\cite{marz2015}.

\subsection{Armazenamento do Master Dataset}

\subsection{Processando Dados}

\subsection{Tecnologias}

\begin{itemize}
  \item Hadoop HDFS
  \item Apache Spark
  \item Apache Thrift
\end{itemize}

\section{SPEED LAYER}

Durante o processamento da camada \textit{batch}, novos dados podem ser
inseridos. Caso somente a \textit{batch} processasse, esses novos dados não
seriam levados em conta em novas \textit{queries}; este problema é resolvido
na Arquitetura Lambda através da camada \textit{speed}, que é capaz de
processar dados em baixa latência \cite{marz2015}.

\subsection{Tecnologias}

\begin{itemize}
  \item Spark
  \item Apache Cassandra
  \item Apache Storm
  \item Kestrel
  \item Apache Kafka
\end{itemize}

\section{SERVING LAYER}

Após os \textit{master datasets} serem criados pela camada \textit{batch}, o
projeto precisa estar pronto para responder as \textit{queries} em baixa
latêcia. Esse papel é da camada \textit{serving}, que indexa e provê interface
para os dados precomputados \cite{marz2015}.

É importante notar que os dados retornados pela camada \textit{serving} estarão
quase sempre desatualizados, pois novos dados chegarão enquanto o
\textit{master dataset} for criado.

\subsection{Tecnologias}

\begin{itemize}
    \item ElephantDB
\end{itemize}
