\chapter[Considerações preliminares]{Considerações preliminares}

\section{Cronograma}

Afim de apresentar uma continuação para este trabalho, a tabela~\ref{tab:atividades_futuras},
apresenta as atividades a serem realizadas e suas respectivas datas.

\begin{table}[h]
\centering
\resizebox{\textwidth}{!}{\begin{tabular}{|l|l|l|}
\hline
\rowcolor[HTML]{EFEFEF}
{\textbf{Atividade}} & {\textbf{Data de Início}} & {\textbf{Data de Término}} \\ \hline
Finalizar o machine learning & 01/01/2016 & 31/01/2016 \\
Criar estratégia determinística / interface da coleta & 01/02/2016 & 29/02/2016 \\
Primeira coleta / analise dos resultados & 01/03/2016 & 31/03/2016 \\
Formalizar resultado da coleta & 01/04/2016 & 30/04/2016 \\
Segunda coleta / teste de hipótese & 01/05/2016 & 31/05/2016 \\
Finalizar documento & 01/06/2016 & 30/06/2016 \\
\hline
\end{tabular}}
\caption{Atividades a serem realizadas}
\label{tab:atividades_futuras}
\end{table}

O objetivo inicial na próxima fase do projeto é finalizar o aprendizado
de máquina. Onde o primeiro passo é fazer a implementação do aprendizado
de máquina, dos testes, e também utilizar da validação cruzada afim de
realizar análises nos resultados das recomendações geradas pelo aprendizado
de máquina.

Após o aprendizado de máquina estar implementado este deve ser adicionado
em uma estratégia de recomendação, ou seja, será criado uma estratégia
no AppRecommender que utiliza o aprendizado de máquina para recomendar
pacotes.

A coleta de dados do usuário já está implementada, porém será adicionado
uma interface gráfica, visando otimizar a interação do usuário com o
software. Após a implementação da interface gráfica será realizado a
coleta de dados dos usuários, e utilizando esses dados será realizado
uma análise a fim de otimizar a estratégia determinística e também
implementar a análise da curva ROC.

Após essa primeira coleta de dados e da análise dos resultados, será
estudado a apresentação dos dados, visualização dos algoritmos e também
verificar se os algoritmos precisam ser adaptados como por exemplo
diminuir o número de atributos usados na técnica de aprendizado de máquina.

Então os resultados da coleta serão formalizados, afim de que o leitor
possa compreender para que servem os dados e o que foi realizado com esses
dados para se chegar aos resultados. Utilizando esses resultados também
será feita uma atualização nas fórumas e nas estratégias de recomendações
criadas, afim de otimizar a recomendação.

Depois de se compreender os dados, os resultados e otimizar a recomendação
será feita uma segunda coleta de dados, onde com os resultados dessa coleta
será realizado o teste de hipótese.

Por ultimo o documento será revisado, analizado e finalizado.
