\chapter{FTP}

O protocolo FTP (\textit{File Transfer Protocol}) foi designado para a transferência de dados entre computadores através e uma rede TCP/IP.

\section{Experimento}
Para análise do tráfego de dados foi utilizada uma rede local Ethernet com CIDR \texttt{192.168.6.0/24}. Foram configurados dois computadores em uma rede local:

\begin{itemize}
	\item Um cliente FTP: Arch Linux, Filezilla, IP \texttt{192.168.6.100};
	\item Um servidor FTP: Debian Linx 8, VSFTPD, IP \texttt{192.168.6.104};
	\item Roteador WRN150, 100mbps;
	\item Cabeamento Ethernet;
	\item Software de captura Wireshark (no cliente).
\end{itemize}

O Wireshark foi configurado para filtrar apenas pacotes entre o cliente e o servidor.

\textbf{Objetivos:} executar login, listar diretório e transferir um arquivo.

\section{Tráfego}
Ao conectar o cliente FTP com o servidor a primeira troca de mensagens é referente ao estabelecimento de uma conexão através do protocolo FTP, utilizando o\textit{three way handshaking}.

Após esse passo o cliente realiza autenticação através da porta 21 (a partir a porta de origem 35580). Após autenticado o cliente envia comandos CWD e PWD  para obter a lista com os arquivos da pasta atual.

Através do comando RETR o cliente pede um arquvo do servidor FTP. Ao aceitar a requisição o cliente abre uma nova conexão com o servidor a partir da porta 39887 (cliente) com destino à porta 26373 (servidor).

O arquivo é transferido com sucesso. Após a transferência o cliente executa o encerramento de sessão com o servidor FTP e, por fim, encerra sua conexão com o protocolo TCP, usado a \textit{flag} FIN e FIN/ACK.
