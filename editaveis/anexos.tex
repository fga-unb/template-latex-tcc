\begin{anexosenv}

\partanexos

\chapter{Term Frequency Inverse Term Frequency}

A técnica \textit{Term Frequency-Inverse Document Frequency (TFIDF)} é uma
medida estátistica usada prioritariamente para indicar a importância de um termo
em dado documento em relação a uma base de documentos \cite{1_rajaraman_ullman_2011}.

Para calcular tal fator, duas etapas distintas são necessárias:

\begin{itemize}
    \item \textbf{\textit{TF: Term Frequency: }} Calcula quantas vezes um termo
    aparece em um dado documento. Dessa forma, seu cálculo pode realizado pela
    seguinte fórmula:

    TF(t) = N(t,d)/T(d)

    Onde: 
    \begin{itemize}
        \item \textbf{N(t, d): } Número de vezes que o termo ``t'' aparece no
        documento ``d''.
        \item \textbf{T(D): } Número de termos presentes no documento ``d''.
    \end{itemize}

    \item \textbf{\textit{IDF: Inverse Document Frequency: }} Usada para diminuir o
    peso de termos muito frequentes em documentos, como os termos ``para'' ou
    ``é''. Para fazer isso, a seguinte fórmula pode ser usada:

    IDF(t) = log(NumDocumentos/D(t))

    Onde:
    \begin{itemize}
        \item \textbf{NumDocumentos: } Quantidade de documentos presentes na
        base.
        \item \textbf{D(T): } Número de documentos que contém o termo ``t''.
    \end{itemize}
\end{itemize}

Com os fatore \textbf{TF} e \textbf{IDF} calculados, basta multiplicar tais
valores para encontar o valor da importância de um termo dado uma base de
documentos. Sendo assim, pode-se entender que o valor do \textit{TFIDF} será
alto para um termo ``t'' se o mesmo é encontrado muitas vezes em um dado
documento e poucas vezes na coleção de documentos.

Existem também outras formas de calcular os valores \textit{TF} e \textit{IDF}.
Essas formas tendem a priorizar certos aspectos em relação ao texto. Uma dessas
abordagens é chamada de \textit{TFIDF-sublinear}. Essa técnica é usada para
atenuar o valor do \textit{TF} para quando um termo é bastante presente em um
documento, mas não é muito presente na base de documentos. A atenuação do valor
do \textit{TF} pode se justificar em casos quando um termo que aparece mais de
uma vez no documento não deva ter um peso substancialmente maior que um termo
que aparece uma única vez. Para calcular o \textit{TF} para este caso, pode-se
usar a seguinte fórmula:

TF-sublinear(t) = 1 + log(TF(t))

Vale ressaltar que outras abordagens também podem ser usadas, como normalização
usando o tamanho médio dos documentos presentes na base de todos os documentos e
\textit{TF} médio entre todos os \textit{TF}s \cite{araujo2011apprecommender}.

\chapter{Segundo Anexo}

Texto do segundo anexo.

\end{anexosenv}

