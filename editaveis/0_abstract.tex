\begin{resumo}[Abstract]
 \begin{otherlanguage*}{english}

     InterSCity is a smart cities platform that has the advantage to focus
     on reuse and interoperability, standing out among the others platforms.
     However, the usage of suitable tools for processing its data is not yet
     present, being an obstacle in the usage of large scale data.
     This work aims then to develop and to design a data processing layer that
     allows InterSCity to handle large scale data, using mainly Big Data, a key
     technology for smart cities.
     With a data processing layer that uses cutting-edge technologies and the
     state-of-art in data processing architectures, it is expected that InterSCity
     achieve what it needs to be the best smart cities platform.

   \vspace{\onelineskip}
 
   \noindent 
   \textbf{Key-words}: smart cities, big data, kappa architecture, lambda architecture
 \end{otherlanguage*}
\end{resumo}
