\chapter[INTERSCITY]{INTERSCITY}

Com o objetivo de prover uma solução para os problemas presentes no ecossistema
de cidades inteligentes, surge o InterSCity, uma plataforma que foca em
aspectos como interoperabilidade, escalabilidade, extensibilidade, e qualidade,
licenciado sob
MPLv2 \footnote{\url{https://www.mozilla.org/en-US/MPL/2.0/}}, e construído
utilizando a arquitetura de microserviços -
MSA \footnote{\url{http://microservices.io/}}.

Baseando-se no desenvolvimento colaborativo e na utilização de projetos software
livre existentes \cite{delesposte2017}, o InterSCity conta atualmente com a
ajuda de diversos colaboradores, e, utilizando metodologias ageis
\cite{delesposte2017}, mantém uma constante evolução ao longo do tempo.

Desenvolvido principalmente em
Ruby on Rails\footnote{\url{http://rubyonrails.org/}},
a plataforma segue padrões que garantem boa extensibilidade e qualidade, e
conta com uma arquitetura madura e bem planejada.

\section{ARQUITETURA}

O InterSCity apresenta uma arquitetura de microserviços distribuida, e pode
ser dividido em quatro componentes principais: o Resource Adaptor, o Resource
Viewer, o Resource Cataloguer e o Data Collector.

Estes quatro componentes são desacoplados entre si, e cada um tem
responsabilidades específicas e bem definidas. A comunicação entre eles é
feita via protocolo AMQP\footnote{\url{https://www.amqp.org/}}, e gerenciada
com a ajuda do projeto RabbitMQ\footnote{\url{https://www.rabbitmq.com/}}. 
