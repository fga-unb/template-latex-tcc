\chapter[INTERSCITY]{INTERSCITY}

Com o objetivo de prover uma solução para os problemas presentes no ecossistema
de cidades inteligentes, surge o InterSCity, uma plataforma que foca em
aspectos como interoperabilidade, escalabilidade, extensibilidade, e qualidade,
licenciado sob
MPLv2 \footnote{\url{https://www.mozilla.org/en-US/MPL/2.0/}}, e construído
utilizando a arquitetura de microserviços -
MSA \footnote{\url{http://microservices.io/}}.

Baseando-se no desenvolvimento colaborativo e na utilização de tecnologias
software livre de ponta \cite{delesposte2017}, o InterSCity conta atualmente
com a ajuda de diversos colaboradores, e, utilizando metodologias ageis
\cite{delesposte2017}, mantém uma constante evolução ao longo do tempo.

Desenvolvido principalmente em
Ruby on Rails\footnote{\url{http://rubyonrails.org/}},
a plataforma segue padrões que garantem extensibilidade e qualidade, e
conta com uma arquitetura madura e bem planejada.

\section{ARQUITETURA}

O InterSCity apresenta uma arquitetura de microserviços distribuida, e pode
ser dividido em quatro componentes principais: o Resource Adaptor, o Resource
Viewer, o Resource Catalog e o Data Collector.

Estes quatro componentes são desacoplados entre si, e cada um tem
responsabilidades específicas e bem definidas. A comunicação entre eles é
feita via RabbitMQ\footnote{\url{https://www.rabbitmq.com/}}, que utiliza o
protocolo AMQP\footnote{\url{https://www.amqp.org/}}.

% desnecessário?
\subsection{Comunicação via passagem de mensagem}

Utilizado principalmente em contextos de concorrência, onde o isolamento é
importante \cite{armstrong2003}, o modelo de comunicação via passagem de
mensagem compõe a ligação entre os diferentes módulos do InterSCity, e é
apresentado por meio do projeto RabbitMQ, que utiliza o protocolo AMQP.

\subsubsection{Termos}
Antes de ver detalhadamente o funcionamento do RabbitMQ no InterSCity, é
interessante o levantamento dos seguintes termos, que serão utilizados
frequentemente:

\begin{itemize}
\item \textbf{Consumidor.} Aplicação que recebe mensagens.
\item \textbf{Produtor.} Aplicação que envia mensagens.
\item \textbf{Conexão.} Compõe o passo inicial da comunicação via passagem
de mensagem, e é iniciada entre uma aplicação e o servidor do
RabbitMQ \cite{amqp2008}. A conexão se mantém aberta durante a execução do
protocolo, e, ao fim da comunicação, esta é fechada.
\item \textbf{Canal.} Meio que permite e isola as passagens de mensagem, e é
atrelado a uma conexão. As mensagens são passadas via canais, de modo que
o consumidor e o fornecedor devem estar ligados ao mesmo canal.
\item \textbf{Fila.} Uma fila de mensagens, atrelada a um canal. A ordem de
chegada e consumo das mensagens é a mesma ordem pela qual as mensagens foram
enviadas \cite{amqp2008}.
\end{itemize}

% desnecessário?
\subsubsection{AMQP}

O AMQP é um protocolo com: funcionalidades modernas, assíncrono, seguro,
portável, e eficiente \cite{amqp2008}. Teve sua criação orientada nos seguintes
requisitos:
%TODO - mexer nos requisitos
(i) ser compacto;
(ii) lidar com mensagens de qualquer tamanho e sem limites;
(iii) permitir em uma mesma conexão diversos canais;
(iv) ter longa vida;
(v) permitir concatenação de comandos de maneira assíncrona;
(vi) ser facilmente modificável para abranger novas necessidades;
(vii) ser retro-compatível;
(viii) ser reparável;
(ix) ser neutro, em relação a linguagens de programação;
(x) ser encarado como um processo.

No InterSCity, o protocolo é utilizado via RabbitMQ, e todos os servidores do
InterSCity funcionam como ambos, consumidor e produtor. 

\subsection{Resource Adaptor}

O microserviço Resource Adaptor é o grande responsável pela comunicação entre
os dispositivos IoT e o InterSCity. Funcionando como um recepcionista durante
as requisições, este módulo gerencia o registro e atualização dos dispositivos
IoT, e redireciona informações importantes dos recursos registrados para os
destinos esperados \cite{delesposte2017}.

Num ciclo de vida típico de um dispositivo, este primeiramente inicia um
pedido de registro como Resource Adaptor, que, gerenciando a conversa com o
módulo Resource Catalog, informa ao recurso seu UUID que será utilizado
internamente na plataforma.

\subsection{Data Collector}

O microserviço Data Collector tem como papel o armazenamento e disponibilização
de dados coletados pelos recursos da plataforma\cite{delesposte2017}.

Os recursos armazenados são então expostos via API, disponibilizando
inclusive dados históricos.

\subsection{Resource Viewer}
Falar sobre o Resource Viewer.

\subsection{Resource Catalog}
Falar sobre o Resource Catalog.
