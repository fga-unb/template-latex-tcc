\chapter[INTERSCITY]{INTERSCITY}

Com o objetivo de prover uma solução para os problemas presentes no ecossistema
de cidades inteligentes, surge o InterSCity, uma plataforma que foca em
aspectos como interoperabilidade, escalabilidade, extensibilidade, e qualidade,
licenciado sob
MPLv2 \footnote{\url{www.mozilla.org/en-US/MPL/2.0/}}, e construído
utilizando a arquitetura de microserviços -
MSA \footnote{\url{microservices.io/}} \cite{delesposte2017}.

Baseando-se no desenvolvimento colaborativo e na utilização de tecnologias
software livre de ponta, o InterSCity conta atualmente com a ajuda de diversos
colaboradores, e, utilizando metodologias ageis, mantém sua evolução ao longo
do tempo \cite{delesposte2017}.

Desenvolvido principalmente em
Ruby on Rails\footnote{\url{rubyonrails.org/}},
a plataforma segue padrões que garantem extensibilidade e qualidade, e
apresenta uma arquitetura madura e bem planejada. O InterSCity se encontra
hospedado no Gitlab \footnote{\url{gitlab.com/smart-city-software-platform}},
onde cada um de seus módulos têm seu repositório.

\section{ARQUITETURA}

O InterSCity apresenta uma arquitetura de microserviços distribuida, e pode
ser dividido em quatro componentes principais: Resource Adaptor, Resource
Viewer, Resource Catalog, e o Data Collector.

Estes quatro componentes são desacoplados entre si, e cada um tem
responsabilidades específicas e bem definidas. A comunicação entre eles é
feita via RabbitMQ\footnote{\url{www.rabbitmq.com/}}, utilizando
o padrão de projeto PubSub \footnote{\url{xmpp.org/extensions/xep-0060.html}},
ocorrendo então a constante troca de mensagens entre os módulos. Utilizado
principalmente em contextos de concorrência, onde o isolamento é
importante \cite{armstrong2003}, o modelo de comunicação via passagem de
mensagem é chave na ligação entre os diferentes módulos do InterSCity.

Num ciclo de vida típico de um dispositivo, este: (i) inicia um pedido de
registro com o Resource Adaptor, que, (ii) gerenciando a conversa com o
módulo Resource Catalog, (iii) informa ao recurso seu UUID, que deverá
ser utilizado internamente deste passo em diante. Após, a comunicação entre o
Resource Adaptor e o dispositivo IoT ainda continuará, mas, (iv) os dados terão
como destino o módulo Data Collector, que armazenará as informações. Por fim,
(vi) as informações contidas no Data Collector serão enviadas para o
Resource Viewer, apresentando então os dados do recurso para o usuário final.
% adicionar imagem?
% preciso referenciar?

\subsection{Microserviços}
\begin{itemize}
    \item \textbf{Resource Adaptor}: é o grande responsável pela comunicação
entre os dispositivos IoT e o InterSCity, e funciona como um recepcionista
durante as requisições, gerenciando o registro e atualização dos
dispositivos IoT, e redirecionando informações importantes dos recursos
registrados para os destinos desejados \cite{delesposte2017}. É escrito em
Ruby on Rails, e, embora converse com os recursos IoT diretamente, comunica-se
com a maior parte dos outros microserviços da plataforma via RabbitMQ.

    \item \textbf{Data Collector}: tem como papel o armazenamento e
disponibilização de dados coletados pelos recursos registrados na plataforma
\cite{delesposte2017}. Também escrito em Ruby on Rails, este microserviço fará
contínua comunicação com a camada de processamento que será desenvolvida.
Constantemente publica dados em tópicos do RabbitMQ, que serão consumidos por
outros microserviços, contudo, a comunicação com o Resource Viewer é feita
diretamente, e não por passagem de mensagem. Os recursos armazenados são
expostos via API, disponibilizando também dados históricos.

    \item \textbf{Resource Viewer}: o unico microserviço da plataforma escrito em
EmberJS\footnote{\url{www.emberjs.com/}}, o Resource Viewer tem o papel de
apresentar os dados armazenados no Data Collector e no Resource Catalog para o
usuário final \cite{delesposte2017}.

    \item \textbf{Resource Catalog}: gerencia o registro de recursos IoT
na plataforma. Este microserviço, que utiliza Ruby on Rails, fornece
identificadores únicos (UUID) para cada recurso que se registra na plataforma
\cite{delesposte2017}.
Esse identificador é utilizado exaustivamente nas futuras comunicações que
ocorrerão com o RabbitMQ.

\end{itemize}
